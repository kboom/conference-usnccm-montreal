\documentclass[c]{beamer}

\usepackage[utf8]{inputenc}
\usepackage{lmodern} 
%\usepackage[scaled=0.8]{beramono}
\usepackage[T1]{fontenc}
\usepackage{microtype}

\usepackage{graphicx}
\usepackage{float}
\usepackage[hypcap]{caption}
\usepackage{subcaption}
\usepackage{algorithm2e}
\usepackage{array}

\usepackage{media9}

\usepackage{mathtools}
\usepackage{amssymb}
\usepackage{amsthm}
\usepackage{amsmath}


\usepackage[utf8]{inputenc}
\usepackage{lmodern} 
%\usepackage[scaled=0.8]{beramono}
\usepackage[T1]{fontenc}
\usepackage{microtype}

\usepackage{graphicx}
\usepackage{float}
\usepackage[hypcap]{caption}
\usepackage{subcaption}
\usepackage{algorithm2e}
\usepackage{array}

\usepackage{media9}

\usepackage{amssymb}
\usepackage{amsthm}
\usepackage{amsmath}


\usepackage{mathtools}

%\usepackage[customcolors]{hf-tikz}
%\usepackage{pgfplots}



%\usepackage{animate}
%\usepackage{movie15}

%\usepackage{epstopdf}

%\epstopdfDeclareGraphicsRule{.gif}{png}{.png}{convert gif:#1 png:\OutputFile}
%\AppendGraphicsExtensions{.gif}



\usepackage[english]{babel}
\selectlanguage{english}

\usetheme{Warsaw}
%\usetheme{CambridgeUS}
\usecolortheme{crane}
\beamertemplatenavigationsymbolsempty

\newcommand{\inred}[1]{\textcolor{red}{#1}}
\newcommand{\inblue}[1]{\textcolor{blue}{#1}}

\beamertemplatenavigationsymbolsempty
\setbeamertemplate{footline}[page number]{}

\captionsetup[figure]{labelformat=empty}


\newcommand{\Prod}[2]{(#1, #2)_{L^2}}
\newcommand{\Mat}[1]{\mathbf{#1}}

\newcommand{\dx}{\, \mathrm{d}x}
\newcommand{\deriv}[2]{\frac{\partial {#1}}{\partial {#2}}}
\newcommand{\Int}[1]{\int_0^1 {#1}\dx}
\newcommand{\Lims}[1]{\left. #1 \right|_0^1}
\newcommand{\IntE}[2]{\int_{#1}{#2}\dx}
\newcommand{\der}[1]{{#1}^\prime}
\newcommand{\Vect}[1]{\mathbf{#1}}

\newcommand{\A}[1]{\Mat{A}\big(#1\big)}
\newcommand{\Bsp}{\mathcal{B}}
\newcommand{\E}[1]{\left[\xi_{#1}, \xi_{#1 + 1}\right]}
\newcommand{\Different}[1]{\textcolor{blue}{#1}}



\title{%
Open source JAVA implementation of the parallel multi-thread alternating direction
isogeometric L2 projections solver for material science simulations}

\author{%
    \inblue{\bf Grzegorz Gurgul (AGH)} \\
    \inblue{Maciej Wo\'{z}niak (AGH)} \\
    \inblue{Marcin \L{}o\'{s} (AGH)} \\
    \inblue{Danuta~Szeliga (AGH)} \\
      \inblue{\bf Maciej Paszy\'{n}ski (AGH)} }
%    \and \\
%  {\bf Collaborators} \\
%   Victor Calo (Curtin University, Perth, Australia) \\
%  Keshav Pingali (ICES, UT, Austin)
%}
\date{{\bf KomPlasTech 2017}, January 15-18, 2017, Zakopane, Poland}

\begin{document}

%%%%%%%%%%%%%%%%%%%%%%%%%%%

\begin{frame}
  \titlepage
\end{frame}

%%%%%%%%%%%%%%%%%%%%%%%%%%%
\begin{frame}{Agenda}

\begin{itemize}
  \item Backgroud
  \item Isogeometric L2 projections algorithm (Maciej Paszy\'{n}ski)
  \item JAVA implementation (Grzegorz Gurgul)
  \item Conclusions
\end{itemize}

\end{frame}

%%%%%%%%%%%%%%%%%%%%%%%%%%%

\begin{frame}{Background}

{\small
\begin{itemize}
  \item Isogeometric L2 projections algorithm
\end{itemize}
  \inblue{Proposed by prof. Victor Calo:} 
L. Gao, V.M. Calo, \emph{Fast Isogeometric Solvers for Explicit Dynamics}, {\bf Computer Methods in Applied Mechanics and Engineering} (2014). 
\begin{itemize}
  \item Applications to time-dependent problems
\end{itemize}
\inblue{Linear elasticity (Fortran sequential ): }M. \L{}o\'{s}, M. Wo\'{z}niak, M. Paszy\'{n}ski, L. Dalcin, V.M. Calo, Dynamics with Matrices Possessing Kronecker Product Structure, {\bf Procedia Computer Science} 51 (2015) 286-295

\inblue{Tumor growth simulations (C++ sequential ): }M. \L{}o\'{s}, M. Paszy\'{n}ski, A. K\l{}usek, W. Dzwinel, Application of fast isogeometric L2 projection solver for tumor simulations, {\bf Computer Methods in Applied Mechanics and Engineering} (2017) 
}

\inblue{Non-linear flow in heterogenous media (Fortran+MPI, parallel ): }
M. Wo\'{z}niak, M. \L{}o\'{s}, M. Paszy\'{n}ski, L. Dalcin, V. Calo, 
Parallel fast isogeometric solvers for explicit dynamics, {\bf Computing and Informatics} (2017) 

\inred{A new open source JAVA code for shared memory machines}
\end{frame}




%%%%%%%%%%%%%%%%%%%%%%%%%%%%%%%%%%%%%%%
%%%%%%%%%%%%%%%%%%%%%%%%%%%%%%%%%%%%%%%

\begin{frame}{Isogeometric L2 projections}


\textbf{In general:} non-stationary problem of the form

\begin{equation*}
  \partial_t u - \mathcal{L}(u) = f(x, t)
\end{equation*}

with some initial state~$u_0$ and boundary conditions
\vspace{2mm}

$\mathcal{L}$ -- well-posed linear spatial partial differential operator

Weak form:
  $\Prod{\partial_t u + \mathcal{L}u}{v} = \Prod{f}{v}$

\vspace{3mm}

Discretization:
\begin{itemize}
  \item spatial discretization: isogeometric finite element method
  \begin{equation*}
  \Prod{\partial_t u_h + \mathcal{L}u_h}{v_h} = \Prod{f}{v_h}
  \end{equation*}
  \\\vspace{2mm}
  $u_h = \sum_i \phi_i$, $v_h\in V_h=span\{\phi_1,\ldots,\phi_n\}$ (B-splines)
  \\\vspace{2mm}
  \item time discretization with explicit method 
  e.g. forward Euler scheme
  \begin{equation*}
   \mathcal{M} u_h^{(t + 1)} =
   \mathcal{M} u_h^{(t)} +
   \Delta t \left(\mathcal{L}u_h^{(t)} + \mathcal{F}\right)
\end{equation*}
\begin{equation*}
  \Prod{u_h^{(t+1)}}{v_h} =  \Prod{u_{h}^{(t)}- \Delta t * \mathcal{L}u_h^{(t)} +\Delta t * \mathcal{F}}{v_h} 
\end{equation*}
  \\\vspace{2mm}
  \item   implies isogeometric L2 projections in every time step 
  
\end{itemize}

\end{frame}



%%%%%%%%%%%%%%%%%%%%%%%%%%%

\begin{frame}{$L^2$ projections -- tensor product basis}

\begin{figure}
  \centering
  \includegraphics[width=0.8\textwidth]{img/2DFEM}
\end{figure}

Isogeometric basis functions:
\begin{itemize}
  \item 1D B-splines basis $B_1(x),\ldots, B_n(x)$
  \item higher dimensions: tensor product basis\\
        $B_{i_1\cdots i_d}(x_1,\ldots,x_d)
        \equiv B^{x_1}_{i_1}(x_1)\cdots B^{x_d}_{i_d}(x_d)$ \\
\end{itemize}
Gram matrix of B-spline basis on 2D domain $\Omega = \Omega_x \times \Omega_y$:

\begin{equation*}
  \begin{aligned}
  \mathcal{M}_{ijkl} &=
  \Prod{B_{ij}}{B_{kl}} =
  \int_\Omega B_{ij}B_{kl}\,\mbox{d}\Omega 
  \end{aligned}
\end{equation*}

Standard multi-frontal solver: $O(N^{1.5})$ in 2D, $O(N^2)$ in 3D

\end{frame}

%%%%%%%%%%%%%%%%%%%%%%%%%%%


\begin{frame}{$L^2$ projections -- tensor product basis}

Isogeometric basis functions:
\begin{itemize}
  \item 1D B-splines basis $B_1(x),\ldots, B_n(x)$
  \item higher dimensions: tensor product basis\\
        $B_{i_1\cdots i_d}(x_1,\ldots,x_d)
        \equiv B^{x_1}_{i_1}(x_1)\cdots B^{x_d}_{i_d}(x_d)$ \\
\end{itemize}
Gram matrix of B-spline basis on 2D domain $\Omega = \Omega_x \times \Omega_y$:
\begin{equation*}
  \begin{aligned}
  \mathcal{M}_{ijkl} &=
  \Prod{B_{ij}}{B_{kl}} =
  \int_\Omega B_{ij}B_{kl}\,\mbox{d}\Omega 
  \end{aligned}
\end{equation*}
 \inblue{\begin{equation*}
  \begin{aligned}
=\int_\Omega B^x_i(x) B^y_j(y) B^x_k(x) B^y_l(y) \,\mbox{d}\Omega \\
  = \int_\Omega (B_i B_k)(x)\,(B_j B_l)(y)\,\mbox{d}\Omega  \\
  = \left(\int_{\Omega_x} B_i B_k \,\mbox{d}x\right)
  \left(\int_{\Omega_y} B_j B_l \,\mbox{d}y\right) \\
  = \mathcal{M}^x_{ik} \mathcal{M}^y_{jl}
  \end{aligned}
\end{equation*}}
\begin{equation*}
\mathcal{M} = \mathcal{M}^x \otimes \mathcal{M}^y \quad\text{(Kronecker product)}
\end{equation*}

\end{frame}


%%%%%%%%%%%%%%%%%%%%%%%%%%%

\begin{frame}[fragile]{Alternating Direction Solver -- 2D}

\begin{figure}
  \centering
  \includegraphics[width=0.4\textwidth]{img/2Dmatrix}
\end{figure}
\begin{equation*}
  \begin{bmatrix}
    A_{11} & A_{12} & \cdots & 0 \\
    A_{21} & A_{22} & \cdots & 0 \\
    \vdots & \vdots & \ddots & \vdots \\
    0 & 0 & \cdots & A_{nn} \\
  \end{bmatrix}
  \begin{bmatrix}
    y_{11} & y_{21} & \cdots & y_{m1}
    \\
    y_{12} & y_{22} & \cdots & y_{m1}
    \\
    \vdots & \vdots & \ddots & \vdots \\
    y_{1n} & y_{2n} & \cdots & y_{mn}
    \\
  \end{bmatrix}
  =
  \begin{bmatrix}
    b_{11} & b_{21} & \cdots & b_{m1} \\
    b_{12} & b_{22} & \cdots & b_{m2} \\
    \vdots & \vdots & \ddots & \vdots \\
    b_{1n} & b_{2n} & \cdots & b_{mn} \\
  \end{bmatrix}
\end{equation*}
\begin{equation*}
  \begin{bmatrix}
    B_{11} & B_{12} & \cdots & 0 \\
    B_{21} & B_{22} & \cdots & 0 \\
    \vdots & \vdots & \ddots & \vdots \\
    0 & 0 & \cdots & B_{mm} \\
  \end{bmatrix}
  \begin{bmatrix}
    x_{11} & \cdots & x_{1n} \\
    x_{21} & \cdots & x_{2n} \\
    \vdots & \ddots & \vdots \\
    x_{m1} & \cdots & x_{mn} \\
  \end{bmatrix}
  =
  \begin{bmatrix}
    y_{11} &
    y_{12} & \cdots &
    y_{1n} \\
    y_{21} & y_{22} & \cdots & y_{2n} \\
    \vdots & \vdots & \ddots & \vdots \\
    y_{m1} &
    y_{m2} & \cdots &
    y_{mn} \\
  \end{bmatrix}
\end{equation*}

\end{frame}


%%%%%%%%%%%%%%%%%%%%%%%%%%%%%%%%%%%%%%%%
%%%%%%%%%%%%%%%%%%%%%%%%%%%

\begin{frame}[fragile]{Alternating Direction Solver -- 2D}
\begin{figure}
  \centering
  \includegraphics[width=1.0\textwidth]{img/RHS}
\end{figure}
\end{frame}
%%%%%%%%%%%%%%%%%%%%%%%%%%%

\begin{frame}{Gram matrix of tensor product basis}

\begin{figure}
  \centering
  \includegraphics[width=0.6\textwidth]{img/Bsplines}
\end{figure}

B-spline basis functions have \textbf{local support} (over $p+1$ elements) 

$\mathcal{M}^x$, $\mathcal{M}^y$, \ldots -- banded structure

$\mathcal{M}^x_{ij} = 0 \iff |i - j| > 2p + 1$

Exemplary basis functions and matrix for cubics

\begin{tiny}
\begin{equation*}
	\begin{bmatrix}
    \Prod{B_1}{B_1} & \Prod{B_1}{B_2} & \Prod{B_1}{B_3} & \Prod{B_1}{B_4} & 0 & 0 & \cdots & 0 \\
    \Prod{B_2}{B_1} & \Prod{B_2}{B_2} & \Prod{B_2}{B_3} & \Prod{B_2}{B_4} & \Prod{B_2}{B_5} & 0 & \cdots & 0 \\
    \Prod{B_3}{B_1} & \Prod{B_3}{B_2} & \Prod{B_3}{B_3} & \Prod{B_3}{B_4} & \Prod{B_3}{B_5} & \Prod{B_3}{B_6} & \cdots & 0 \\
    \vdots & \vdots & \vdots & \vdots &  \vdots & \vdots &  & \vdots\\
    0 & 0 & \ldots & \Prod{B_n}{B_{n-3}}& \Prod{B_n}{B_{n-2}} & \Prod{B_n}{B_{n-1}} & \Prod{B_n}{B_n}
  \end{bmatrix}
\end{equation*}
\end{tiny}

\end{frame}




%%%%%%%%%%%%%%%%%%%%%%%%%%%

\begin{frame}[fragile]{Alternating Direction Solver -- 2D}

{Two steps} -- solving systems with $\Mat{A}$ and $\Mat{B}$ in different \emph{directions}
\begin{equation*}
  \begin{bmatrix}
    A_{11} & A_{12} & \cdots & 0 \\
    A_{21} & A_{22} & \cdots & 0 \\
    \vdots & \vdots & \ddots & \vdots \\
    0 & 0 & \cdots & A_{nn} \\
  \end{bmatrix}
  \begin{bmatrix}
    y_{11} & y_{21} & \cdots & y_{m1}
    \\
    y_{12} & y_{22} & \cdots & y_{m1}
    \\
    \vdots & \vdots & \ddots & \vdots \\
    y_{1n} & y_{2n} & \cdots & y_{mn}
    \\
  \end{bmatrix}
  =
  \begin{bmatrix}
    b_{11} & b_{21} & \cdots & b_{m1} \\
    b_{12} & b_{22} & \cdots & b_{m2} \\
    \vdots & \vdots & \ddots & \vdots \\
    b_{1n} & b_{2n} & \cdots & b_{mn} \\
  \end{bmatrix}
\end{equation*}
\begin{equation*}
  \begin{bmatrix}
    B_{11} & B_{12} & \cdots & 0 \\
    B_{21} & B_{22} & \cdots & 0 \\
    \vdots & \vdots & \ddots & \vdots \\
    0 & 0 & \cdots & B_{mm} \\
  \end{bmatrix}
  \begin{bmatrix}
    x_{11} & \cdots & x_{1n} \\
    x_{21} & \cdots & x_{2n} \\
    \vdots & \ddots & \vdots \\
    x_{m1} & \cdots & x_{mn} \\
  \end{bmatrix}
  =
  \begin{bmatrix}
    y_{11} &
    y_{12} & \cdots &
    y_{1n} \\
    y_{21} & y_{22} & \cdots & y_{2n} \\
    \vdots & \vdots & \ddots & \vdots \\
    y_{m1} &
    y_{m2} & \cdots &
    y_{mn} \\
  \end{bmatrix}
\end{equation*}

%$\Mat{A}$, $\Mat{B}$ -- multidiagonal matrices for one-dimensional bases \\
Two one dimensional problems with multiple RHS:
\begin{itemize}
\item $ n \times n $ with $m$ right hand sides $\rightarrow$ $O(n*m)=O(N)$
\item $ m \times m $ with $n$ right hand sides $\rightarrow$ $O(m*n)=O(N)$
\end{itemize}
Linear computational cost $O(N)$

\end{frame}



%%%%%%%%%%%%%%%%%%%%%%%%%%%%%%%%%%%%%%%%
%
%\begin{frame}{Isogeometric L2 projections}
%
%\inblue{The computational cost of the solver is so low,\\  that most of the time is spent on the integration} 
%
%\begin{columns}
%  \begin{column}{0.5\textwidth}
%  \begin{figure}
%      \centering
%      \includegraphics[width=1.1\textwidth]{img/Fraction}
%      \caption{Time spent on integration with respect to time spent on factorization (below 1 percent of the total time for 2D problems, \\ for all $p$ and $N$)}
%    \end{figure}
%  \end{column}
%
%  \begin{column}{0.5\textwidth}
%    \begin{figure}
%      \centering
%      \includegraphics[width=0.95\textwidth]{img/speedup_p3}
%      \caption{Speedup of parallel integration with GALOIS \\cubics, 2D problem \\ different mesh sizes}
%    \end{figure}
%  \end{column}
%\end{columns}
%
%\inblue{Expensive isogeometric integration that can be speeded-up \\ on multi-core machines } \\
%\end{frame}
%%%%%%%%%%%%%%%%%%%%%%%%%%%


\begin{frame}{Hitting elastic material (1/2)}

    \begin{figure}
      \includegraphics[width=0.95\textwidth]{img/Figure4}
      \caption{Snapshoots from the simulation }
    \end{figure}


\end{frame}

\begin{frame}{Hitting elastic material (2/2)}

    \begin{figure}
      \includegraphics[width=0.95\textwidth]{img/Figure5}
      \caption{Snapshoots from the simulation }
    \end{figure}


\end{frame}



%%%%%%%%%%%%%%%%%%%%%%%%%%%

\begin{frame}{JAVA implementation (1/10)}

TO DO

\end{frame}
%%%%%%%%%%%%%%%%%%%%%%%%%%%

\begin{frame}{Conclusions and future research}


Our research is funded by Polish National Science Centre \\ grant no. DEC-2015/19/B/ST8/01064

\end{frame}

%%%%%%%%%%%%%%%%%%%%%%%%%%%
%%%%%%%%%%%%%%%%%%%%%%%%%%%

\end{document}
